%----------------------------------------------------------------------------------------
%   PACKAGES AND OTHER DOCUMENT CONFIGURATIONS
%----------------------------------------------------------------------------------------

\documentclass[11pt]{article}

\usepackage[english]{babel}
\usepackage[utf8x]{inputenc}
\usepackage{amsmath}
\usepackage{graphicx}
\usepackage{csquotes}
\usepackage{fancyvrb}
\usepackage{array}
\usepackage{relsize}
\usepackage[a4paper, total={6in, 8in}]{geometry}
\usepackage[colorinlistoftodos]{todonotes}

\newenvironment{conditions}
  {\par\vspace{\abovedisplayskip}\noindent\begin{tabular}{>{$}l<{$} @{${}={}$} l}}
  {\end{tabular}\par\vspace{\belowdisplayskip}}
\newcommand{\shortrnote}[1]{ &  & \text{\footnotesize\llap{#1}}}

%----------------------------------------------------------------------------------------
%   HEADING
%----------------------------------------------------------------------------------------

\newcommand{\BigO}[1]{\ensuremath{\operatorname{O}\left(#1\right)}}

\title{\textsc{Computational Finance}\\Coursework}
\author{Lawrence Jones \{lmj112\}}

\date{}
\begin{document}
\maketitle

%----------------------------------------------------------------------------------------
%   EXERCISE 1
%----------------------------------------------------------------------------------------

\subsection*{Exercise 1. Dividend Discount Model}


Assume that each stock of a company pays a dividend $D_t$ at the end of every
year, $t = 1, 2, 3, \ldots$, and denote by $r > 0$ the constant annual interest
rate (use yearly compounding).

\subsubsection*{Part (a)}

Computing the Net Present Value of the dividend stream:

\begin{equation}
  \mathlarger{\mathlarger{\sum}}_{k=1}^{\infty}
    \frac{(1 + g)^{k}~D_{0}}{(1+r)^{k}}
\end{equation}
\
where:

\begin{conditions}
    D_{0}           & Initial dividend amount \\
    g               & Dividend growth rate \\
    r               & Constant annual interest rate
\end{conditions}

% Remove D0 from summation
\begin{flalign}
  && &= D_{0} \left(
    \mathlarger{\mathlarger{\sum}}_{k=1}^{\infty}
      \frac{(1 + g)^{k}}{(1+r)^{k}}
    \right) \shortrnote{Extract $D_{0}$} \\
\
  && &= D_{0} \left(
    \mathlarger{\mathlarger{\sum}}_{k=0}^{\infty}
      \frac{(1 + g)^{k}}{(1+r)^{k}}
      -
    1
  \right) \shortrnote{Extend summation limits} \\
\
  && &= D_{0} \left(
    \frac{1}{1 - \frac{1 + g}{1 + r}}
      -
    1
  \right) \shortrnote{Summation to $\infty$} \\
\
  && &= D_{0} \left(
  \frac{1 + r}{r - g}
    -
  \frac{r - g}{r - g}
  \right) \shortrnote{Multiply by $\frac{1+r}{1+r}$} \\
\
  && &= D_{0} \left(
  \frac{1 + g}{r - g}
  \right)
\end{flalign}

\end{document}
