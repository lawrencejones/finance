%----------------------------------------------------------------------------------------
%   PACKAGES AND OTHER DOCUMENT CONFIGURATIONS
%----------------------------------------------------------------------------------------

\documentclass[11pt]{article}

\usepackage[english]{babel}
\usepackage[utf8x]{inputenc}
\usepackage{amsmath}
\usepackage{graphicx}
\usepackage{csquotes}
\usepackage{fancyvrb}
\usepackage{array}
\usepackage{relsize}
\usepackage[a4paper, total={6in, 8in}]{geometry}
\usepackage[colorinlistoftodos]{todonotes}

\setlength{\parindent}{0pt}

\newenvironment{conditions}
  {\par\vspace{\abovedisplayskip}\noindent\begin{tabular}{>{$}l<{$} @{${}={}$} l}}
  {\end{tabular}\par\vspace{\belowdisplayskip}}
\newcommand{\shortrnote}[1]{ &  & \text{\footnotesize\llap{#1}}}

%----------------------------------------------------------------------------------------
%   HEADING
%----------------------------------------------------------------------------------------

\newcommand{\BigO}[1]{\ensuremath{\operatorname{O}\left(#1\right)}}

\title{\textsc{Computational Finance}\\Coursework}
\author{Lawrence Jones \{lmj112\}}

\date{}
\begin{document}
\maketitle

%----------------------------------------------------------------------------------------
%   EXERCISE 1
%----------------------------------------------------------------------------------------

\subsection*{Exercise 1. Dividend Discount Model}

Assume that each stock of a company pays a dividend $D_t$ at the end of every
year, $t = 1, 2, 3, \ldots$, and denote by $r > 0$ the constant annual interest
rate (use yearly compounding).

\subsubsection*{Part (a)}

Computing the Net Present Value of the dividend stream:

\setcounter{equation}{0}
\begin{equation}
  \mathlarger{\mathlarger{\sum}}_{k=1}^{\infty}
    \frac{(1 + g)^{k}~D_{1}}{(1+r)^{k}}
\end{equation}
\
where:

\begin{conditions}
    D_{1}           & Initial dividend amount \\
    g               & Dividend growth rate \\
    r               & Constant annual interest rate
\end{conditions}

\setcounter{equation}{0}
\begin{eqnarray}
  &&= D_{1}\left(\sum_{k=1}^{\infty} \frac{\left(1+g\right)^{k-1}}{\left(1+r\right)^{k}}\right) \\
  &&= \frac{D_{1}}{\left(1+g\right)} \left(\sum_{k=1}^{\infty} \frac{\left(1+g\right)^{k}}{\left(1+r\right)^{k}}\right) \\
  &&= \frac{D_{1}}{\left(1+g\right)} \left(\sum_{k=0}^{\infty} \frac{\left(1+g\right)^{k}}{\left(1+r\right)^{k}} - 1\right) \\
  &&= \frac{D_{1}}{\left(1+g\right)} \left(\frac{1}{1 - \left(\frac{1+g}{1+r}\right)} - 1\right) \\
  &&= \frac{D_{1}}{\left(1+g\right)} \left(\frac{1+r}{r-g} - 1\right) \\
  &&= \frac{D_{1}}{\left(1+g\right)} \frac{1+g}{r-g} \\
  &&= \frac{D_{1}}{r-g}
\end{eqnarray}

\subsubsection*{Part (b)}

Find expression for\dots

\setcounter{equation}{0}
\begin{eqnarray}
  & \sum_{t=1}^{\infty} tx^{t} &= x + 2x^{2} + 3x^{3} + \dots \\
                               &&= \sum_{t=1}^{\infty} \left( \sum_{k=t}^{\infty} x^{k} \right) \\
                               &&= \sum_{t=1}^{\infty} \left( \frac{x^{t}}{1 - x} \right) \\
                               &&= \frac{\frac{x}{1 - x}}{1 - x} \\
                               &&= x\left(1 - x\right)^{-2}
\end{eqnarray}

The net present value of this dividend stream can be represented as\dots

\begin{eqnarray}
  & \sum_{t=1}^{\infty} \frac{\left(D_{1} - I\right) + tI}{\left(1 + r\right)^{t}} \
  &= \sum_{t=1}^{\infty} \frac{\left(D_{1} - I\right)}{\left(1+r\right)^{t}} + \sum_{t=1}^{\infty} \frac{tI}{\left(1+r\right)^{t}} \\
  &&= \left(D_{1} - I\right)\left( \sum_{t=0}^{\infty} \frac{1}{\left(1+r\right)^{t}} - 1 \right) + \dots \\
  &&= \left(D_{1} - I\right)\left( \frac{1}{1 - \left(\frac{1}{1+r}\right)} -1 \right) + \dots \\
  &&= \left(D_{1} - I\right)\left( \frac{r+1}{r} -1 \right) + \dots \\
  &&= \left( \frac{D_{1} - I}{r} \right) + \sum_{t=1}^{\infty} \frac{tI}{\left(1+r\right)^{t}} \\
  &&= \dots + I\left( \frac{1}{1+r} \left(1 - \frac{1}{1+r}\right)^{-2} \right) \\
  &&= \dots + I\left( \frac{1}{1+r} \left( \frac{1}{1 - \frac{1}{1+r}} \right)^{2} \right) \\
  &&= \left( \frac{D_{1} - I}{r} \right) + I\left( \frac{1+r}{r^{2}} \right) \\[4mm]
  &&= \frac{rD_{1} + I}{r^{2}}
\end{eqnarray}

\newpage

%----------------------------------------------------------------------------------------
%   EXERCISE 2
%----------------------------------------------------------------------------------------

\subsection*{Exercise 2. Fixed Income Securities}

Consider a 10\% coupon bond with a face value of £100 that matures in 2 years
from now. Assume that there is one coupon payment per year and that the bond is
absolutely risk free.

\subsubsection*{Part (a)}

The cash flow stream with spot rates is computed in the following table.

\begin{center}
  \begin{tabular}{| l | l | l | l |}
    \hline
    \textbf{Year}           & 1     & 2     & 3      \\ \hline
    \textbf{Spot Rate}      & 4\%   & 6\%   & 8\%    \\ \hline
    \textbf{Discount}       & 0.962 & 0.890 & 0.794  \\ \hline
    \textbf{Cash Flow}      & 10    & 110   & 0      \\ \hline
    \textbf{Present Value}  & 9.62  & 97.90 & 0      \\ \hline
  \end{tabular}
\end{center}

The price of the bond is the sum of the present values\dots

\[\sum_{t=1}^{2} PV_{t} = 9.62 + 97.90 + 0 = £107.52\]

\subsubsection*{Part (b)}

From our definition of Yield to Maturity with yearly coupon payments\dots

\setcounter{equation}{0}
\begin{equation}
P = \frac{F}{\left(1 + \lambda\right)^{n}} + \sum_{k=1}^{n} \frac{C}{\left(1 + \lambda\right)^{k}}
\end{equation}
\
where:

\begin{conditions}
  P               & Bond price (market value) \\
  F               & Face value \\
  n               & Years until maturity \\
  \lambda         & Yield to Maturity \\
  C               & Coupon amount (coupon rate * face value)
\end{conditions}

\begin{eqnarray}
  & P \
\
  &= \frac{F}{\left(1 + \lambda\right)^{2}} + \frac{C}{1 + \lambda} + \frac{C}{\left(1 + \lambda\right)^{2}} \\
  &&= \frac{F + C}{\left(1 + \lambda\right)^{2}} + \frac{C\left(1 + \lambda\right)}{\left(1 + \lambda\right)^{2}} \\
  &&= \frac{F + 2C + C\lambda}{\left(1 + \lambda\right)^{2}} \\
\end{eqnarray}

Rearranging for a $\lambda$ quadratic gives\dots

\[P\lambda^{2} + \left(2P - C\right)\lambda + \left(P - F - 2C\right) = 0\]

Using quadratic formula\dots

\begin{eqnarray}
  & \lambda &= \frac{ -\left(2P - C\right) \pm \sqrt{\left(2P - C\right)^{2} - 4P\left(P - F - 2C\right)} }{2P} \\
  &&= \frac{ C - 2P \pm \sqrt{C^{2} + 4CP + 4FP} }{2P}
\end{eqnarray}

For this bond\dots

\[ \lambda = -1.966 \text{~or~} 0.059 \]

Yield is positive, so the yield value is 5.9\%.

\subsubsection*{Part (c)}

Definition of Macaulay duration\dots

\[ D = \frac{PV[t_{0}]t_{0} + \dots + PV[t_{n}]t_{n}}{PV[t_{0}] + \dots + PV[t_{n}]} \]
\
where:

\begin{conditions}
  D               & Macaulay duration \\
  t_{i}           & Time in years \\
  PV[t_{i}]       & Present value of cash flow at time $t_{i}$ wrt. yield
\end{conditions}

So for this bond\dots

\setcounter{equation}{0}
\begin{eqnarray}
  & D &= \frac{PV[t_{0}]t_{0} + PV[t_{1}]t_{1} + PV[t_{2}]t_{2}}{PV[t_{0}] + PV[t_{1}] + PV[t_{2}]} \\
  &&= \frac{0 + 9.44 + 2*98.08}{0 + 9.44 + 98.08} \\
  &&= \text{1.91 years}
\end{eqnarray}

\subsubsection*{Part (d)}

First we are required to compute the spot rate from the next year. Due to expectation dynamics,
we know that the spot rates to come can be calculated from todays forward rates.

\[ s' = f_{1,2}, f_{1,3}, \dots \]

We have the definition for a forward rate between time i and j\dots

\[ f_{i,j} = \left[ \frac{\left(1 + s_{j}\right)^{j}}{\left(1 + s_{i}\right)^{i}} \right]^{\frac{1}{j - i}} - 1 \]

So we must use this to define $s'$\dots

\begin{eqnarray}
  & s'_{1} = f_{1,2} &= \left[ \frac{\left(1 + 0.06\right)^{2}}{1.04} \right]^{\frac{1}{1}} - 1 = 8.04\% \\
  & s'_{2} = f_{1,3} &= \left[ \frac{\left(1 + 0.08\right)^{3}}{1.04} \right]^{\frac{1}{2}} - 1 = 10.06\%
\end{eqnarray}

We can now calculate the price of the bond in a years time, noting that the first coupon payment is still due\dots

\[ P = \left(10 / 1\right) + \left(110 / 1.0804\right) = £111.81 \]
\\
From this we can now compute Yield to Maturity\dots

\begin{eqnarray}
  & P &= \frac{F}{\left(1 + \lambda\right)^{n}} + \sum_{k=0}^{n} \frac{C}{\left(1 + \lambda\right)^{k}} \\[4mm]
  &&= \frac{F}{\left(1 + \lambda\right)^{1}} + \frac{C}{\left(1 + \lambda\right)^{0}} + \frac{C}{\left(1 + \lambda\right)^{1}} \\[4mm]
  &&= \frac{F + C}{1 + \lambda} + C \\[4mm]
  & \lambda &= \frac{F + C}{P - C} - 1 \\[4mm]
  &&= 8.04\%
\end{eqnarray}

\newpage

%----------------------------------------------------------------------------------------
%   EXERCISE 4
%----------------------------------------------------------------------------------------

\subsection*{Exercise 4. Fixed Income Securities}

Assume a constant ideal bank with yearly interest of 6\%, compounded monthly. Consider an
annuity that pays £100 monthly over a period of ten years; compute the size of a single cash
flow arising after five years that is equivalent to the annuity.

\subsubsection*{Part (a)}

We calculate the annuity present value as the sum of all the payments at the discount factor
for the time they are received\dots

\setcounter{equation}{0}
\begin{eqnarray}
  & PV[annuity] &= \sum_{t=1}^{10 \times 12} \frac{100}{\left(1 + \frac{0.06}{12}\right)^{t}} \\[4mm]
  &&= 100 \left[ \frac{\left(1 - 1.005^{-120}\right)}{1.005\left(1 - 1.005\right)} \right] \\[4mm]
  &&= £9,007.35
\end{eqnarray}

A cash flow stream of equivalent value is one that has an equivalent present value. We must find
the amount ($a$) of the single payment which when discounted for five years of compounded interest
is equivalent to the present value of the annuity\dots

\begin{eqnarray}
  & PV[annuity] &= \frac{a}{\left(1 + \frac{0.06}{12}\right)^{5 \times 12}} \\[4mm]
  & a &= PV[annuity] \times 1.005^{60} \\[4mm]
      &&= £12,149.57
\end{eqnarray}

\end{document}
